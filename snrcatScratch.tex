\section{Scratch}
has the spatial and spectral sensitivity to resolve

good spatial and spectral 


talk about how much "power" comes from the higher energy  range,

EGRET point source sensitivity is ${\rm \sim1x10^{-7}}\flux{}$
 \url{http://fermi.gsfc.nasa.gov/science/instruments/table1-1.html} get this number from somewhere else


\jamie{somewhere I should also say that identifying an extended source as such gives better spectra, see 2FHL}

\cite{Thomson93}, egret observed a total of 1.5 million celestial \gam{}

\jamie{ Fermi goals: 1. Resolve the \gam~sky: the origins of diffuse emission and the nature of unidentified sources: Source identification through good source localization, measurement of spectra across broad energy range, nearly continuous monitoring of the sky for temporal variability
	2. Understand the mechanisms of particle acceleration in celestial sources:}

\jamie{understanding accel plus identification of the potential SNRs motivates the SNR cat }

\jamie{snrs as source of Galactic cosmic rays, and thus as drivers of Galactic evolution}

-what's to gain from studyng the population vs individual?

-constraint on CR energy density in the Galaxy

-do the observed g-ray properties match those
of models?

-like index etc?

\jamie{Another assumption to speed things up is that the PSF doesn't vary too much with event incidence angle in individual bins. To ensure this even more, events with a reconstructed angle > 66.4deg (cos theta = 0.4) are removed (idk why this angle)}

This section is about the tools (and intricacies there in) developed to study SNRs with Fermi

The addSrcs framework extends/exploits pointlike 

addSrcs in this section or merged into the next one?

I need to stress that this is not just a tool developed, but there's an art and finesse to it (talk about the problems we overcame), there was much iteration.

tool plus knowledge to get the best result

what are the aspects of addSrcs tht needed knowhow, finagling

, \ts{} map creation and extension fitting.
\jamie{assume I've mentioned \ts{} maps already}  For these reasons, we utilized \ptlike{} to perform an analysis of 

we endeavored to take 

First something about the SNR cat, and wanting to uniformly study the regions around SNRs, maybe this is a new section
to facilitate the characterization of the sky

what are some motivations for various addSrcs decisions 

How to distinguish addSrcs from how addSrcs was applied to the SNR cat. For the SNRcat, it was addSrcs in PS mode with specifics relevant to the SNR cat needs (like what?)


tested pipeline with 6 sources (I have the tests somewhere, see old Fermi symp poster, Gal Evo too)