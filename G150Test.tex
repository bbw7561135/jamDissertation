\graphicspath{ {Figures/G150} } 
\chapter{SNR G150.3+4.5}
\label{chap:G150}
test
\section{Introduction} 
\jamie{not ready yet} Supernova remnants have long been thought to be the primary accelerators of cosmic rays up to the knee of the cosmic-ray energy spectrum. 

something about the benefit of Pass 8 extending the viable LAT energy range for analysis to TeV energies and what that affords us in closing the gap between GeV and TeV.

Something about LAT being all sky and easier to detect broadly extended sources than for TeV?

2FHL blindly detected

faint radio \citep{Gao14}, \cite{Gerbrandt14}

what to say about radio SNRs? Connect CRs to nonthermal emission and the LAT and 
Something about SNRs, cosmic ray accelerators, radio detections, connection between radio-LAT observations, G150 detection, 2FHL blind detection and SNRs at TeV (all young?)

We describe the LAT and analysis results in $\S$\ref{sec:LATobs}, detail
multiwavelength observations in $\S$\ref{sec:Multiwave}, and discuss various emission origin scenarios in $\S$\ref{sec:Discuss}.
%%%%%%%%%%%%%%%%%%%%%%%%%%%%%%%%%%%%%%%%%%%%%%%%%%%%%%%%%%%%%%%%
%
%         FermiLat  Observations and  Analysis 
%
%%%%%%%%%%%%%%%%%%%%%%%%%%%%%%%%%%%%%%%%%%%%%%%%%%%%%%%%%%%%%%%%
\section{\FermiLat  Observations and  Analysis }\label{sec:LATobs}
\subsection{Data Set and Reduction}\label{sec:LATdata}
\FermiLat is a pair conversion telescope sensitive to high energy \gam{}s  from 20 MeV to greater than 1 TeV \citep{2FHL}, operating primarily in a sky-survey mode which views  the entire sky every 3 hours. The LAT has a wide field of view ($\sim$2.4 sr), a large effective area of $\sim$8200 cm$^2$ above 1 GeV for on axis events and a  68\% containment radius angular resolution  of $\sim$0.8$^\circ$  at 1 GeV. For further details  on the instrument and its performance see \cite{atwood09} and \cite{lat_perf}.

In this analysis, we  analyzed 7 years of Pass 8 data, from August 2nd 2008  to August 2nd 2015. The Pass 8 event reconstruction provides a significantly improved angular resolution \jamie{this is sadly unimportant unless I'm at higher energy or using the PSF types. The P8 total PSF at 1 GeV is about the same as for P7REP. It's the acceptance/effective area that are considerably better at this energy}, acceptance, and background event rejection \citep{atwood13b,atwood13}, all of which lead to an increase in the effective energy range and sensitivity of the LAT. Source class events were analyzed within a 14$^\circ$x14$^\circ$ region centered on SNR \Gone~using the P8R2\_SOURCE\_V6 instrument response functions, with a pixel size of 0.1$^{\circ}$. To reduce contamination from earth limb \gam{}s, only events with zenith angle less than 100$^{\circ}$ were included.

For spectral and spatial analysis we utilized both the standard \Fermi{} Science Tools (version 10-01-01)\footnote[1]{\url{http://fermi.gsfc.nasa.gov/ssc/}}, and the binned maximum likelihood package \ptlike~\citep{Kerr10}. \ptlike~provides methods for simultaneously fitting the spectrum, position, and spatial extension of a source, and was extensively validated in \cite{Lande12}. Both packages fit a source model, the Galactic diffuse emission, and an isotropic component (which accounts for the background of misclassified charged particles and the extragalactic diffuse \gam{}  background)\footnote[2]{\url{http://fermi.gsfc.nasa.gov/ssc/data/access/lat/BackgroundModels.html}} to the observations. In this analysis, we used the standard Galactic diffuse ring-hybrid model scaled for Pass 8 analysis, gll{\_}iem{\_}v06.fits (modulated by a power law function with free index and normalization), and for the isotropic emission,  we used iso{\_}P8R2{\_}SOURCE{\_}V6{\_}v06.txt, extrapolated to 2 TeV as in \cite{2FHL}.

In our source model for the region, we included sources from the third \FermiLat catalog \citep[3FGL]{3FGL} within 15$^\circ$ of the center of our region of interest (RoI). We replaced the position and spectrum of any 3FGL pulsars in the region with their corresponding counterpart  from the LAT 2nd pulsar catalog \citep{2PC}.  Residual emission unaccounted for by 3FGL sources is present in the RoI due to the increased time range and different energy selection with respect to that in 3FGL. We added to the RoI several significant (${\rm TS \geq 16}$) point sources to account for this unmodeled emission and minimize the global residuals. The closest of these sources added was about 1$^{\circ}$ away from the edge of the best fit GeV disk. \jamie{ Considering the size of the PSF at 1 GeV, the affect of these sources on the disk fit was assumed to be  negligible. do I need to say more about these sources? should I mention adding them automatically and iteratively based on TS maps and reference SNRcat/2FHL?}.  The normalization and spectral index of sources within 5$^{\circ}$ of the center of the RoI were free to vary, whereas all other source parameters were fixed. A preliminary maximum likelihood fit of the RoI was performed, and  sources with a test statistic (TS) $<$ 9 (TS is defined as,  ${\rm TS}=2~{\rm Log}(\mathcal{L}_1 / \mathcal{L}_0)$ where $\mathcal{L}_1$ 
 is the likelihood of source plus background and  $\mathcal{L}_0$ that of just the background) were removed from the model. 

\subsection{Morphological Analysis}\label{sec:LATmorph}
Studying the spatial extension of sources with the LAT is non-trivial due to the energy-dependent point spread function (PSF) and strong diffuse emission present in the Galactic plane. Soft spectrum point sources and uncertainties in the diffuse model can act as sources of systematic error when not accurately modeling extended emission as such, particularly at low energies where the PSF is broad. To strike a balance between the best angular resolution and minimal source and diffuse contamination, we restrict our morphological analysis to energies between 1 GeV and 1 TeV. We divide this energy range into 12 \jamie{4bpd} logarithmically spaced bins for both \ptlike~and \gtlike~binned likelihood analyses. 

Three  unidentified 3FGL sources are located within the extent of \Gone. 3FGL J0425.8+ 5600, located approximately 0.6$^\circ$ from the center of the SNR, is the closest of the three sources and is described with a power law spectrum of index ${\rm \Gamma = 2.35\pm 0.17}$  in the 3FGL catalog. The closest radio source to 3FGL J0425.8+5600 is NVSS J042719+560823, at 0.25° away (Ref?). 3FGL J0423.5+5442, exhibits a power law spectral index, ${\rm \Gamma = 2.63\pm 0.15}$, with no clear multiwavelength source association. Finally, \psrLike{} has a pulsar-like spectrum, yet in a timing survey performed with the 100-m  Effelsberg radio telescope, \cite{Barr13} were unable to detect pulsations from the source down to a limiting flux density of $\sim$ 0.1 mJy. This source is located about 0.84$^{\circ}$ from the center of the SNR. We discuss \psrLike{} and potential association with \Gone{} further in $\S$\ref{sec:Dist}). 

In our analysis, we removed 3FGL J0425.8+5600 and 3FGL J0423.5+544 from the RoI, but kept \psrLike{} in the model since preliminary analyses showed clear positive residual emission at the position of the source if it was removed from the RoI. Figure \ref{fig:1GeV_resTSmap} shows a residual TS map for the region around \Gone. This point source detection-significance map was created by placing a point source modeled with a power law of photon index $\Gamma$ = 2  at each pixel and gives the significance of detecting a point source at each location above the background. 

%put file name in {} to get it to compile with dots in the name!
%for png, have to use pdfchain
\begin{figure}[!ht]
	\begin{centering}
		\includegraphics[width=\columnwidth]{{G150_1GeV_resTsmap_radio_2FHL_noLabs}.pdf}
		\includegraphics[width=\columnwidth]{{G150_1GeV_resTsmapNoG150_radio_2FHL_noLabs}.pdf}
		\caption{Background subtracted residual TS map above 1 GeV with 0.1$^\circ$x 0.1$^\circ$ pixels, centered on SNR \Gone. The orange circle and translucent shading show the fit disk radius and 1$\sigma$ errors, respectively, for the extended source, the orange cross shows the position of \psrLike{} (included in the background model), blue dashed circle is the extent of the radio SNR, and white dashed circle depicts \ghard{}. Bottom map includes \Gone{} in the background model, top does not.
			\label{fig:1GeV_resTSmap}}
	\end{centering}
\end{figure}

We modeled the excess emission in the direction of \Gone ~with a uniform intensity, radially-symmetric disk, simultaneously fitting the spatial and spectral components of the model  via \ptlike. The extension of the disk was initialized with a seed radius of $\sigma$ = 0.1$^\circ$ and position centered on the radio position of \Gone. We define the significance of extension as in \cite{Lande12}; ${\rm TS_{ext} = 2~log(\mathcal{L}_{ext} / \mathcal{L}_{ps})}$, with $\mathcal{L}_{ext}$ being the likelihood of the model with the extended source and $\mathcal{L}_{ps}$ that of a point source located at the peak of emission interior to the extended source. For the disk model we found that  ${\rm TS_{ext} = 298}$, for the best fit radius, ${\rm \sigma = 1.40^\circ \pm 0.03^\circ}$, and position,  ${\rm R.A. = 55.46^\circ \pm 0.03^\circ }$, ${\rm DEC. = 66.91^\circ \pm 0.03^\circ }$, all in excellent agreement with the radio SNR size and centroid determined in \cite{Gao14}. \jamie{do other LAT papers give the TS of extended source too? ${\rm TS = 373}$}. We tried adding back in to our model the two removed 3FGL sources but both were insignificant when fit on top of the best fit disk. The bottom map in Figure \ref{fig:1GeV_resTSmap} is a residual TS map of the same region as the top map of the same Figure, but with the disk source included in the background model, demonstrating that the disk can account well for the emission in the region and justifying the exclusion of the two aforementioned 3FGL sources.

The morphology of the radio emission is suggestive of an elliptical or ring morphology, so both of these spatial models were tested as well. For the ring model, the fit reduced to a disk with parameters matching those stated above. Using the elliptical model showed a weak improvement over the radially symmetric model at the 2.6$\sigma$ level (${\rm \Delta TS = 9}$ with two additional degrees of freedom), which we did not consider significant enough to say the GeV emission had an elliptical morphology (see Table \ref{tab:LATres}). For the remainder of this study, we only considered the disk spatial model.
\jamie{ double check this for 1GeV- 1TeV. I was done for 1-562 GeV, running it now!}

\ghard{} is the extended source detected in the 2FHL catalog found to be overlapping the northern region of \Gone{} \cite{2FHL}. The source has a power law spectral index ${\rm \Gamma = 1.66 \pm 0.2}$, and disk radius ${\rm \sigma = 1.27^\circ \pm 0.04^\circ}$ (see Figure \ref{fig:1GeV_resTSmap}). When comparing the best fit extension of the 2FHL source with the result from this paper, factoring in the uncertainty in both extension and position, we see that the $>$ 50 GeV and $>$ 1 GeV results are not incompatible. It is likely that the paucity of events above 50 GeV is the cause of the smaller fit radius, as opposed to the difference arising from the effects of an energy dependent morphology. To explore the connection between the 2FHL and above 1 GeV emission, we tested a few other spatial hypotheses.

First, we  replaced the ${\rm \sigma = 1.40^{\circ}}$ disk with an another disk matching the spectral and spatial parameters of \ghard{} and calculated the likelihood with this new source's position and extension fixed. For this hypothesis, we find ${\rm TS_{ext} =  165}$, and  ${\rm TS = 226}$, demonstrating that the fixed disk matching the 2FHL source is clearly disfavored over the previously determined best fit disk at this energy. Our next test consisted of placing a second extended source on top of the best fit disk detected above 1 GeV. We added a source, initially matching the spatial and spectral parameters of \ghard{}, to our source model of the region (in addition to the ${\rm \sigma = 1.40^{\circ}}$ disk), and fit its spectrum and extension. Fitting a second extended source in this region serves two purposes: 1. it acts as a check on whether there was residual emission unaccounted for by the previously best-fit disk, and 2. it allows us to determine if the best fit disk can be split into two spectrally distinct, components. This fit resulted in the source wandering north (but still partially overlapping \Gone{}) and having an insignificant extension, ${\rm TS_{ext} =  4}$. Details on the spatial parameters are given in Table \ref{tab:LATres}.


\jamie{Something about J0426?like  how modeling G150 as point vs extended if it's really extended can affect the fit of other point sources nearby, like J0426, so show the spectrum of this source too?  I fit both the norm and index of the source. Save this for discussion? How does the spectrum of J0426 change with the new source? Maybe the most that needs to be said is that below 1 GeV it's confused with }

\jamie{from Josh's paper:  modelling the spectrum of an intrisically extended source as point sources skews the PS spectrum to softer energies "Specifically, modeling a spatially extended source as point-like will systematically soften measured spectra", idk if I get why. We see it with the 2 removed 3FGL sources being softer than what the disk winds up being.}

\begin{deluxetable}{cccccccc}
\setlength{\tabcolsep}{0.04in}
\tablewidth{0pt}
\tabletypesize{\scriptsize}
\tablecaption{LAT Analysis Results\label{tab:LATres}}
\tablehead{
\colhead{Spatial Model\,\tablenotemark{a}} & 
\colhead{TS} & 
\colhead{TS$_{ext}$} &
\colhead{$\sigma$} &
\colhead{Association} &
\colhead{Class} &
\colhead{Spatial model} &
\colhead{Extension [deg]} \\
\colhead{} & 
\colhead{} & 
\colhead{} &
\colhead{$^\circ$} &
\colhead{} &
\colhead{} &
\colhead{} &
\colhead{}}
\startdata
J0526.6$-$6825e      &    278.843 &    -32.850 & 49.80  & LMC                & gal    & 2D Gaussian         & 1.87 \\
J0617.2+2234e        &    189.048 &      3.033 & 398.64 & IC~443             & snr    & 2D Gaussian         & 0.27 \\
J0822.6$-$4250e      &    260.317 &  -3.277 &  63.87 & Puppis A       & snr    & ...               & 0.37 \\
%\cutinhead{Thee Point Sources}
%\sidehead{Uniform Disk}

\enddata
\tablecomments{~This is mostly from 2FHL, just playing with the table. Don't need a table for just disk hypothesis, but maybe to have disk 3  point sources compatison. I tried adding more sources on top of the 3FGL sources and there's no significant residual. where to say something about testing searching for point sources overlapping the extended source and trying to fit an extended source on top as well? in this table give the disk model with best spectral spatial params, TS, TSext  dof, LL, then the model with just the 3 3FGL sources (no disk) spectrum of each, TS, dof + LL( didn't relocalize the se sources), separate spatial spectral tables?
}
\tablenotetext{a}{comments and notes?}
\end{deluxetable}


\begin{figure}[!ht]
	\begin{centering}
		\includegraphics[width=\columnwidth]{{G150_extProf_1GeV}.png}
		\includegraphics[width=\columnwidth]{{TSextVsEn_G150_1GeV_1TeV_sigma1_4}.png}
		\caption{Not sure I want to include these, replot, get rid of titles, and make them look nicer if I do want to include. Top plot shows that the TS peaks at the best fit extension. Lower plot gives a sense of how significant the extension is (vs a point source) across the analyzed energy range. If I keep, add text in the section
			\label{fig:extProf}}
	\end{centering}
\end{figure}

\begin{figure}[!ht]
	\begin{centering}
		\includegraphics[width=\columnwidth]{{G150.3+4.5_RadInt}.png}
		\caption{Include this to show that there's significant emission above the background? Replot without the point model.
			\label{fig:radInt}}
	\end{centering}
\end{figure}

\begin{figure}[!ht]
	\begin{centering}
		\includegraphics[width=\columnwidth]{{G150.3+4.5_sources}.png}
		\includegraphics[width=\columnwidth]{{G150.3+4.5_source}.png}
		\caption{Should I  include (diffuse subtracted) counts map of the region  to show what the actual counts (not just TS map) like and where the 3FGL sources are? (redo these removing the extended source and including the 3FGL? or 3FGL removed in a diffetent color? Not sure I need these and the TS maps. Redo them as pdf/eps. Remove titles, use same cmap as figure 1, bigger bolder font.
			\label{fig:1GeV_cmaps}}
	\end{centering}
\end{figure}
%%%%%%

\subsection{Spectral Analysis}\label{sec:LATspec}
After determining the best fit morphology with \ptlike{} for the GeV emission coincident with SNR \Gone{}, we used those results as a starting point for our \gtlike{} maximum-likelihood fit of the region to estimate the best spectral parameters for our model. The LAT data is well described by a power law from 1 GeV to 1 TeV with a photon index, ${\rm \Gamma = 1.82 \pm 0.04}$, and energy flux above 1 GeV of ${\rm (7.17 \pm 0.73)~ x 10^{-11}~ erg~cm^{-2}~ s^{-1}}$and TS = 389 \jamie{pointlike results were index = 1.80 flux = ${\rm (7.17 \pm 0.73~ x 10^{-11})~ erg~cm^{-2}~ s^{-1}}$}. We tested the \gam{} spectrum of the extended disk for spectral curvature using a log-normal model (Log Parabola), and find no significant deviation from a power law (${\rm \Delta TS \sim 1}$). Figure \ref{fig:G150SED} shows the best-fit power law spectral energy distribution for the GeV source whose morphology was described in Section $\ref{sec:LATmorph}$. Spectral data points were obtained by dividing the energy range into 12 logarithmically spaced bins and modeling the source with a power law of fixed spectra index, ${\rm \Gamma = 2}$.

\jamie{what else to include here? Systematics. Bracketing IRFs, alt iem, try varying the extension? still to be done.}

%%%G150 pointlike SED: G150_1GeV_resTsmap_radio_noLabs.pdf
\begin{figure}[!ht]
	\begin{centering}
		\includegraphics[width=\columnwidth]{Figures/{G150.3+4.5_gtlikeNewSED}.pdf}
		\caption{Spectral energy distribution for the extended source coincident with SNR \Gone{} from 1 GeV to 1 TeV. Red line corresponds to the best fit power law model. Crosses are shown with with statistical error bars \jamie{add systematics when I have them}.
			 \jamie{Butterfly?}
			\label{fig:G150SED}}
	\end{centering}
\end{figure}

%pointlike SED G150.3+4.5_3FGLJ0426.7+5437_SED.png
\begin{figure}[!ht]
	\begin{centering}
		\includegraphics[width=\columnwidth]{Figures/{3FGL_J0426.7+5437_gtlike_defaultBkgRefit_SED}.pdf}
		\caption{Spectral energy distribution of \psrLike{}. \jamie{Replot this make it look nicer}
			\label{fig:J0426SED}}
	\end{centering}
\end{figure}


%%%%%%%%%%%%%%%%%%%%%%%%%%%%%%%%%%%%%%%%%%%%%%%%%%%%%%%%%%%%%%%%
%
%         Multiwavelength  Observations and  Analysis 
%
%%%%%%%%%%%%%%%%%%%%%%%%%%%%%%%%%%%%%%%%%%%%%%%%%%%%%%%%%%%%%%%%

\section{Multiwavelength  Observations and  Analysis }\label{sec:Multiwave}
\subsection{HI}\label{sec:HI}
\subsection{CO?}\label{sec:CO}
Jack's looking into Planck data for HI and CO
\subsection{X-ray}\label{sec:Xray}
No diffuse nonthermal X-ray emission observed by ROSAT. No point sources near the center? Should a  pulsar even  be near the center? How to quantify this? Can we place a limit on ambient density with an upper limit on thermal X-ray emission? Magnetic filed with nonthermal? upper limit on potential pulsar spin-down power, then see what fraction of that power the lum of J0426 would be, assuming it's at the distance of G150 to see if that's reasonable it being the putative pulsar?

%%%%%%%%%%%%%%%%%%%%%%%%%%%%%%%%%%%%%%%%%%%%%%%%%%%%%%%%%%%%%%%%
%
%         Discussion and Results
%
%%%%%%%%%%%%%%%%%%%%%%%%%%%%%%%%%%%%%%%%%%%%%%%%%%%%%%%%%%%%%%%%
\section{Discussion and Results}\label{sec:Discuss}
Idk about subsections here, but I have them everywhere else so maybe it's weird to not have them here?
\subsection{SNR or PWN?}\label{sec:PWNvsSNR}

The follow-up observations of the \gam{} emission in the direction of \Gone{}, presented here, of the source detected above 50 GeV in 2FHL have led to the detection of an extended \gam{} source whose centroid and radius match extremely well with those of the radio detected SNR. The broad size of the extended source and correlation with the radio shell leave few plausible scenarios for the nature of the GeV emission. \jamie{We argue here that an SNR is favored over a pulsar wind nebulae (PWN) as the generator of the observed \gam{}s.}

One potential scenario is that the \gam{}s in this region are produced by a pulsar wind nebula (PWN) generated by the putative puslar of SNR \Gone{}. The first problem with the PWN hypothesis is that there is no pulsar candidate \jamie{(be it radio, X-ray or \gam{})} detected near the centroid of the SNR to power a PWN. While 3FGL J0425.8+ 5600 is the closest \gam{} source to the center of the remnant, it does not have a pulsar-esque spectrum, it lies about ${\rm 0.25^\circ }$ away, and we showed in $\S$\ref{sec:LATmorph} that with the best-fit disk hypothesis, neither 3FGL J0425.8+ 5600 nor 3FGL J0423.5+5442 are significant in the likelihood model of the region. \psrLike{}, with a spectrum reminiscent of a pulsar,  may actually be one, but as discussed previously, \cite{Barr13} detect no pulsations from the source. Furthermore, the source is $0.84^\circ$ away from the centroid of \Gone{}. Typical pulsar ballistic velocities range from ${\rm V_{PSR} \sim 400-500~km~s^{-1}}$, with extreme velocities exceeding 1000 ${\rm km~s^{-1}}$ \citep{Gaensler06}. If \psrLike{} was the compact remnant of the progenitor star that birthed \Gone{}, it would have to be traveling with a velocity, ${\rm V_{PSR} = 1125  km~s^{-1}}$, making it one of the fastest known pulsars \cite{Chatterjee05} \jamie{Fastest pulsar (till 2011 at least) 1100 km/s, more recent ref?}] \jamie{this assumes the closest distance and youngest age}. Move things around because I haven't discussed distance and age yet? Or maybe that should be done in the HI section? At least distance should be

There's nothing that looks like a PWN in the direction of the \psrLike{}

something about nondetection of xray emission, upper limit in X? GeV SED index don't look like PWN? Not sure this is true.

Despite these reasons, a PWN origin can not be ruled out, due to lack of an associated pulsar.


\jamie{it being extended means there aren't many options for what it can be. Lining up so well the \Gone{} lends to thinking it's the SNR, maybe PWN? talk about why Gao, Gebradt don't htink it's the pwn}

Why not PWN, radio emission looks more ring/shell like. radio and gamma index are suggestive of SNR, not PWN (is this true?). GeV matches with radio. no pulsar detected. LAT PWN have broad plateau spectrum, do they extened across the whole energy band? Can I say what edot of the psr would be if I know the LAT flux and xray flux? 

- here's the flux detected from G150

- assuming the derived distance, here's the luminosity

- If it's a PWN does this luminosity suggest a spindown power?
-or at least what fraction of some reasonable spin down power is this lum?

- If we have an upper limit on the x-ray flux, does the ration of x-ray to gamma suggest a spin down power?

-which paper I was looking at today mentioned the connection between xray lum and psr spin down power? W41 parer does something, but I thought there was another?

I haven't been calling the GeV source \Gone{} this whole time. should I be?

Size + HI suggest that near distance corresponding to different HI velocities suggest it's aged, spectrum looks more like young SNR (hard + no GeV break ). Is it a weird young remnant or weird aged one? Leptonic dominated if young, hadronic dominated if older? Something about nearby dense clouds masking hadronic emission? Maybe this is only true for MeV cosmic rays that are screened out though and it would only mask the pion bump, but not this higher energy emission?

PWN or SNR. Can we rule out PWN? See W41 paper, MSH 11-61A, Fabios recent G326 work (no, he just tries to use the PSF types and testing different model templates to try to disentangle SNR from PWN)?


No PSR candidate near center (should it be near the center? Depends on age)
Is there some limit we can place on the PWN based on not seeing the pulsar? Like on Edot? OR something like Mattana et al. 2009 correlation between  $\mathrm{flux_x / flux_g \propto}$   Edot? 

Assume it's in Sedov phase based on size + near distance, and calculate age, upper limit on Edot base on lack of x-ray flux? Or maybe if I assume the sources is the PWN and GeV radius is PWN radius, then can I estimate Edot based on size and evolution inside  SNR?

If we assume close distance, age is only $\approx$ 5kyr, maybe this is a transitional SNR? What do others like this look like? Puppis A? Gamma Cygni is a similar age too.something 

Say something about what the radio index is and the connection to the gev index
\jamie{remake the G150 all SNR SEDs with the new gtlike data points}

Discuss how it compares to other LAT SNRs
\subsection{Distance Considerations}\label{sec:Dist}
Estimate distance in HI section, do age here?
\subsection{Nonthermal Modeling}\label{sec:naima}

SNR shock fronts are known accelerators of cosmic rays to very high energies. There are potentially multiple radiation mechanisms operating at the shock that produce GeV \gam{}s. Accelerated electrons can give rise to inverse Compton (IC) emission via upscattering of ambient cosmic microwave background (CMB), stellar, and IR photon fields, as well as non-thermal bremsstrahlung radiation. Energetic protons can collide with ambient protons in the surrounding, producing neutral pions which decay into \gam{} photons. 

To infer the properties of the underlying relativistic particle populations in the SNR environment, it is vital to understand the origin of the observed \gam{} emission detected from  \Gone{}. To do so, we employ the \nai{} Python package. \nai{} is an open-source code base that computes the non-thermal radiation from a relativistic particle population \citep{Zabalza15}. It utilizes known parameterizations and analytic approximations to the various non-thermal processes (i.e., synchrotron, IC, bremsstrahlung, and pion decay emission), which results in the calculations being computationally inexpensive. \nai{} also makes use of \emc{}, a Markov chain Monte Carlo (MCMC) ensemble sampler for Bayesian parameter estimation \citep{Foreman13}. The sampler is used to find the best-fit parameters of the radiative models to the observed photon SED for a given particle distribution function. 

To determine the best fit parameters, \nai{} calls \emc{} to sample the log-likelihood function (i.e.,  the likelihood of the observed data given the assumed spectrum) of the radiative model. The radiative models require as input a particle distribution function to model the present-age electron or proton spectrum. \nai{} inherently assumes a one-zone, homogeneous distribution, and we scaled the likelihood function by a uniform prior probability distribution. For this work, we model the particle spectra as power laws with an exponential cut off,

\begin{equation}
{\rm \frac{dN}{dE}_{e,p} = A_{e,p} (E / E_0) ^{-\alpha} \exp {\bigg ( }\frac{- E}{ E_{cutoff}}{\bigg)}}
\end{equation}
where E is the particle energy, ${\rm E_0}$ the reference energy, $\alpha$ the spectral index, and ${\rm E_{cutoff}}$ the cutoff energy. The electron distribution's normalization is related to the proton normalization through the electron-to-proton ratio scaling factor, $A_e = K_{ep} A_p$. We also assume that the electron and proton distributions have the same spectral shape.

In our spectral model, we assume a gas density, ${\rm n_0 = 1~cm^{-3}}$ for proton-proton \jamie{and bremss when I get it working} interactions For IC emission, we include CMB  
Talk about free/ fixed params of the model, reference the table, and figure to show best fit, discuss results and what the fits imply regarding lep/had dom and energy in e- p.

Used radio SED from \citep{Gerbrandt14}


\begin{deluxetable}{cccccc}
\setlength{\tabcolsep}{0.04in}
\tablewidth{0pt}
\tabletypesize{\scriptsize}
\tcap{Naima Model Best Fit Parameters\label{tab:naima}}
\tablehead{
\colhead{s} & 
\colhead{${\rm K_{ep}}$} & 
\colhead{${\rm A_p}$ } &
\colhead{B\tablenotemark{a}} & 
\colhead{${\rm E_{cutoff(e)}}$} &
\colhead{${\rm E_{cutoff (p)}}$}} %\\
%\colhead{} & 
%\colhead{} & 
%\colhead{${\rm 10^48}~TeV^{-1}$  } &
%\colhead{} &
%\colhead{} &
%\colhead{}
\startdata
\sidehead{\underline{Fixed ${\rm K_{ep}} = 0.01$}}
1.5 $\pm$ 0.2     &    0.01 &     -32.850 &  49.80  & LMC   & 2 \\ 

\sidehead{\underline{Fixed ${\rm K_{ep}} = 0.1$}}
1.5 $\pm$ 0.2    &    0.1 &     -3.277  &  63.87  & Puppis A    & 2    \\

\sidehead{\underline{Fixed ${\rm K_{ep}} =1$}}
1.5 $\pm$ 0.2    &    1 &     -3.277  &  63.87  & Puppis A      & 2 \\

\sidehead{\underline{Fixed s}}
2 $\pm$ 0.2    &    1 &     -3.277  &  63.87  & Puppis A      & 2 \\
%\sidehead{Uniform Disk}

\enddata
\tablecomments{~ \jamie{add better caption} Results from naima model?  Right now the free params are index, kep, eleccut, protcut, B. Fixed are nh, all the IC photon field values distance (this is just for determining flux) \jamie{Correct values aren't in yet}\jamie{}add units to params}

\tablenotetext{a}{Calculated in \gtlike}
\end{deluxetable}



\jamie{should I include what the like function looks like here?}

\jamie{estimates min of negative loglike through sampling}	

\jamie{for thesis, I can get into more details about what the radiation models are doing and what the MCMC sampler does?}

\jamie{naima to do:Add bremss. properly scale the radio flux density. should I be using a power law particle dist function because I have a power law photons spec? increase walker number burn in and runs?Do the energetics make sense? Are there params that I can fix? Kep, B? Any reason to have different n? Liz had doubts about the pp component extending to such high energy, is this really an issue? gtlike finds the best fit index, does the fit value from naima for the particle dist match this? maybe I should fix the particle index based on this? Should the cutoff be the same for e- and p? Do I have to set E0 in naima to 10 GeV for Kep? LAT RCW 86 uses 512 keV for elec, 1 GeV for PP for min energies and says Kep = 0.01 at 1 GeV/c. Try fixing Kep to 1, 0.1, 0.01, index to 2}

Discuss implications of the naima fits. Do they show preference for lep/had? suggest something about total energy content in particles?

for synchrotron $\alpha = (1-s)/2$, where $\alpha$ is the radio spectral index, and s the electron distribution power law index. Same for IC below break?

SNR cat figure 8 suggests there are only 4(ish) SNRs with an index < 2

eastern shell (Jack called this overall) radio index $\alpha = 0.4 \pm 0.17$  \cite{Gao14}, but $\alpha = 0.69 \pm 0.24$ for the western

For energies below the high energy break
For pion and bremss $\Gamma = 2\alpha + 1$ (says SNR cat)

For IC, $\Gamma = \alpha + 1 $ ) for positive $\alpha$

From \cite{Gaensler06} Typical indices for PWNe are $\sim -0.3 \lesssim \alpha  \lesssim  0$ in the radio band, and ($\Gamma \approx 2$) in the X-ray band. So $\alpha$ is not inconsistent, but at the boundary.

For puppis A paper, why did they use particle index = gam photon index?

cr abundances at earth kep = 0.01  (Hillas 2005).

Sooo, my index is consistent with either?
\begin{figure}[!ht]
	\begin{centering}
		\includegraphics[width=\columnwidth]{Figures/G150_ICsync_PP_SED.pdf}
		\caption{ \Gone. Naima SED \jamie{bigger font, get rid of text for each line, better colors}}
			\label{fig:naimaSED}
	\end{centering}
\end{figure}

Another section here? The last thing I want to do is say something about what further observations are necessary to get at any unanswered questions. 

Deeper x-ray observations to search for the compact stellar remnant. 

What can be done in TeV? The difficulty pointed TeV observations are that it might be difficulty for them to detect such broadly extended emission (why? I know they'd have to tile their observations, but there's something inherently difficulty for them about observing large extended sources. Is it just that the emission is spread out so it might be faint and below the detection threshold?). What about HAWC? Why does it not detect the emission that Fermi clearly shows extending to VHE energies? Is it not as sensitive at this energy for some reason? HAWC energy range extends to 100 GeV.


%%%%%%%%%%%%%%%%%%%%%%%%%%%%%%%%%%%%%%%%%%%%%%%%%%%%%%%%%%%%%%%%
%
%         Conclusion
%
%%%%%%%%%%%%%%%%%%%%%%%%%%%%%%%%%%%%%%%%%%%%%%%%%%%%%%%%%%%%%%%%
\section{Conclusions}\label{sec:Conc}
We analyzed 7 years of \FermiLat{} data in the direction of SNR \Gone{}, lowering the energy threshold from that previously reported in the 2FHL catalog, and report detection of significantly extended \gam{} emission coincident with the entirety of the radio remnant's shell. We find the emission from 1 GeV to 1 TeV to be well described by a power law of spectral index ${\Gamma = \rm 1.82 \pm 0.04}$, with  morphology consistent with a uniform disk with best-fit radius, {\rm $\sigma = 1.40^{\circ} \pm 0.03^{\circ}$}. Something about why we think it's the SNR and what we get from X-ray analysis. To estimate the distance to the SNR, we obtained  an HI spectrum toward \Gone{} from the Leiden/Argentine/Bonn survey of Galactic HI. Calculating distances from the derived HI velocity peaks, we showed that the most reasonable distance estimate places \Gone{} at a distance of ${\rm d = 0.4~kpc}$, making it one of the closest known SNRs detected by the LAT \jamie{how off can this number be? SNR catalog says 8 SNRs are withing 1.5 kpc and have some kind of classification in the catalog. These are (closest first) Vela, cygnus loop , Vela Jr, RX J1713, G073, S147, IC443, Monoceros loop, of 0.4 kpc is correct for G150, it's the second closest LAT detected SNR. Even at 1.5 kpc it would be within top 10}. Using this distance and a standard Sedov-Taylor SNR evolution model, we estimate the age of the \Gone{} to be ${\rm T \sim 5~kyr}$ \jamie{same as dist, how much can this be off? and what does that mean for assumptions of what phase of evolution the SNR is in}. To assess the underlying particle population acting in \Gone{} we use the \nai{} Python package to fit the observed radio and \gam{} SED to non-thermal electron and proton radiation models. We find that blah, which suggests more blah. Something about how G150  fits in with other LAT detected SNRs based on age, spectrum. End with what  further observations can get us.