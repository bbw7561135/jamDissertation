\chapter{Dissertation Overview}
\label{chap:intro}

We begin Chapter \ref{chap:gamAstr} of this dissertation with a brief discussion of the history of \gam{} astronomy, followed by a description of the mechanisms which lead to astrophysical \gam{}s, and the sources that produce them. We conclude with a history of \gam{} observations leading up to those by the current space-borne\mev{} to \tev{} \gam{} observatory, the \Fermi{} Gamma-Ray Space Telescope (hereafter, \Fermi{}).

In Chapter \ref{chap:FGST}, we present the \Fermi{} telescope's design and primary instrument, the \lat{}. We then describe the \lat{}'s performance and capabilities. 

Chapter \ref{chap:Rems} is a discussion of the  astrophysical theory of \snrs{}, connection between \snrs{} and \crs{}, as well as the historical and current state of \snr{} observations. We conclude with a discussion of the present state of \gam{} studies of \snrs{}.

In Chapter \ref{chap:snrcat}, we present new work (published as \cite{snrCat}) on a systematic study of the population of radio \snrs{} emitting\gev{} \gam{}s, as detected by the \lat{}. We also present a new tool and analysis method, initially developed for this study and extended for general \lat{} analysis.

Chapter \ref{chap:2FHL} is a presentation of the  work published as \cite{2FHL}. We applied this new analysis tool to study the all-sky population of hard Galactic \gam{} sources, from an energy of 50\gev{} to 2\tev{}, with a focus on detection of spatially extended sources.

In Chapter \ref{chap:G150}, we present a new study, in preparation for publication, where we performed an in-depth follow-up analysis of one of the newly discovered spatially-extended sources detected in\gam{}s for the first time via our analysis in Chapter \ref{chap:2FHL}. 

Finally, in Chapter \ref{chap:conc}, we summarize the results of this thesis and reflect on potential avenues to further something something\jamie{fill this in}.



