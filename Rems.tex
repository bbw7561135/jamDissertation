\chapter{Supernova Remnants: Theory and  Observation}
\label{chap:Rems}

\section{Introduction}\label{Rems:intro}
Why study SNRs, what are they and what do we learn about them that's unique to \gam 's and the LAT?

\jamie{Maybe this section should really be the first one? Like  talk about SNRs, the thing we're studying before how we study it?}
What came before? What did EGRET find.

One of the primary goals of fermi was to identify these sources and the site of CR acceleration. But also to uniquely open this high energy window that where no other telescopes really operated
Brief history of radio detection of SNRs. detection at other wavelengths. what we see at \g-rays?

Motivation for why to study them
\citep{Sturner95}
\citep{Esposito96}
\section{Supernova Remnant Evolution/Theory}

History of SNR,  radio detections, then x-ray, can't really ID in new ones with the LAT, theory: snow plow , ST evolution, radiative phase, shell type, mixed morphology (shell plus center filled thermal x-ray) 
\section{Pulsar Wind Nebulae \jamie{get rid of /minimize this section}}

How much of my thesis is really about pwn? in 2FHL we detect some pwn. If including above 10gev work, they'll be there too. Much of the thesis is really about extended gamma-ray sources, but not sure how that fits into the title and chapters yet

Do I need to get into composite SNRs (composite means SNR + PWN ) Maybe relevant for G150? Some things about interaction of reverse shock with PWN and crushing/reverberations of the PWN?

How we detect gamma-rays from SNRs/PWNe in the Galaxy leads to and analysis section maybe?

\section{Summary}\label{Rems:summ} In this section we summarized the end phase of stellar evolution (just enough to motivate SNRs) and descried the environs surrounding the supernova; development and phases of \glspl{snr} (and \glspl{pwn}?).  In particular we detailed the nonthermal emission mechanisms that produce \g-ray radiation, detection of young vs middle-aged( evolved, interacting with surroundings/dense medium),TeV detects younger typically, the troubles of detecting extension from them(?) something about different emission zones? Troubles disentangling hadronic from leptonic at \g-rays. \g-ray spectral and morphological features. Trends across the population wrt spectral shape/breaks, higher luminosity for interacting rems. Cosmic rays, using gamma-rays to probe CR population. So much of \g-ray astro is really about studying CRs, how much to say about them? 

\section{Scratch}
This chapter needs a different title. It's more focused on the specific sources being studied in this thesis. Galactic extended sources, SNRs, PWNe, but as in the SNRcat, not just extended SNRs, point-like SNRs as well.

Less focus on PWNe. Only give as much as I feel I need to support mentioning them a bit for 2FHL?

The focus of this section is supernova remnants in a gamma-ray context. Theory of evolution, what the gamma-ray emission is like, what we can learn from them individually.  This leads to the 1st SNR cat section for what we can do with them ensemble
