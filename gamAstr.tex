\chapter{Gamma-ray Astronomy }
\label{chap:gamAstr}

\begin{figure}[h!]%[t] 
	\centering
	\makebox[\linewidth]{\includegraphics[width=1.\columnwidth]{Figures/panel_hulk001b}}
\end{figure}

%\begin{quote}
%	``Once there was a boy, and the boy loved stars very much'' 
%	\begin{center}---by Oliver Jeffers, from \it{How to Catch a Star} \end{center}
%\end{quote}


%ugh commentes
%\section{Introduction}\label{gamAstr:intro}


\section{\gam{} Emission Mechanisms }\label{gamAstr:Emiss}
The story of \gam{}s from astrophysical objects is a tale of the most extreme, energetic, and violent environments in our universe. \gam{}s are the highest named energy of light, starting from around several hundred\kev{} and extending up, with the highest energy \gam{}s detected being hundreds of\tev{}. Aside from the  nuclear fission, \gam{}s are produced solely by charged particles being accelerated to\gev{} and\tev{} energies and interacting with a target particle, be it other matter or photon fields.  Below we summarize the various non-thermal emission mechanisms giving rise to \gam{}s, namely, \ic{} radiation, non-thermal \brems{} (both of an accelerated electron, or leptonic, origin), and neutral pion decay emission (of an accelerated proton, or hadronic origin). We also described the \sync{} radiation process. While not typically observed up to \gam{} energies (with \sync{}  emission from the Crab nebula being an exception \cite{AbdoCrab}, with the tail of the emission extending to hundreds of\mev{}), \sync{} photons plays a significant role in understanding \ic{} \gam{} emission since the processes both arise from a shared, underlying electron population. Furthermore, observations of \sync{} emission at radio and X-ray energies is vital in  constraining a source's underlying charged particle population and the resultant \gam{} source spectra. We follow \cite{Houck06} for many of the photon emissivities given below.

\subsection{Synchrotron Radiation}\label{gamAstr:sync}
When a relativistic electron moves in a magnetic field, it experiences a force perpendicular to its velocity which causes the electron to accelerate and travel in a helical path around the magnetic field lines. This acceleration results in radiation of photons, referred to as \sync{} radiation \cite{Blumenthal70,Pacholczyk70,Rybicki86,Longair11}. The total power emitted at a frequency $\nu$ from a relativistic electron (Lorentz factor, $\gamma \gg 1$) spiraling in a magnetic field, B is:

\begin{equation}\label{eq:syncPow}
{\rm P_{emitted}(\nu) = 
\frac{\sqrt{3} q^3 B \sin\alpha}{m_e c^2} F(\nu/\nu_c) }
\end{equation}
where q is the electron's charge,  ${\rm m_e}$ is the electron's rest mass, c is the speed of light,  ${\rm \alpha }$ is the angle between the magnetic field vector and the electron's velocity vector (the pitch angle), F is called the first synchrotron function and is an integral over an irregular modified Bessel function (see \cite{Rybicki86}), and ${\rm \nu_c}$ is the frequency at which most of the power is radiated. This so-called critical frequency is:

\begin{equation}\label{eq:nuCrit}
\nu_c = \frac{3q B \gamma^2}{4\pi m_e c} 
\sin\alpha \equiv \nu_0 \gamma^2 \sin\alpha
\end{equation}
We note that the power emitted is inversely proportional to the mass, which explains the insignificance of proton synchrotron radiation, because the rest mass of the proton is $\sim$ 2000 times greater than that of the electron.

For an isotropic distribution of electrons we define  ${\rm N(p,\alpha)}$ as the number of electrons per unit solid angle and momentum with momentum p and pitch angle $\alpha$. \cite{Houck06} show that the differential photon-emissivity spectrum (\ie{}\ number of radiated photons per unit time per unit energy) is:

\begin{equation}
\frac{d n}{d \omega d t} =
\frac{\sqrt{3}q^3 B}{h m_e c^2 \nu}
\int
N(p)
R \left(\frac{\omega}{\omega_0 \gamma^2}\right)dp
\end{equation}
with $\omega$ being the radiated photon energy, ${\rm N(p) = 4\pi N(p,\alpha)}$ for an isotropic pitch-angle distribution, and
\begin{equation}
R(x) \equiv \frac{1}{2} \int_0^\pi
\sin^2 \alpha
F\left(\frac{x}{\sin\alpha}\right) d \alpha 
\end{equation}


\subsection{Non-Thermal \brems{}}\label{gamAstr:bremss}
Bremsstrahlung radiation occurs when charged particles (electron-electron or electron-ion in this case) pass near each other, causing the  primary charge to decelerate, and emit a photon. As noted by \cite{Haug75}, the electron-electron \brems{} system has no electric dipole moment, and it is the quardrapole moment that dominates in the non-relativistic regime, and thus for low energies, the electron-electron contribution is negligible. This situation changes for relativistic energies where the cross section of electron-electron \brems{} is comparable to the electron-ion cross section (where the ratio is $\simeq 0.86$ for photon energies above 10\mev{} \citep{Baring99}), thus for \gam{} studies, we include the non-thermal electron-electron component in the total \brems{} emission.

For a differential spectrum, ${\rm N_e(p)}$, corresponding to the accelerated electrons, the total emissivity is  the sum of the electron-electron and electron-ion \brems{}. The combined differential photon-emissivity spectrum is: s

\begin{equation}
{\rm \frac{d n}{d \omega d t} =
	n_e \int
	N_e(p) v_e
	\frac{d \sigma_{e e}}{d \omega}dp +
	n_Z \int
	N_e(p) v_e
	\frac{d \sigma_{e Z}}{d \omega}dp}
\end{equation}
where ${\rm d \sigma_{e Z}/d \omega}$ and ${\rm d \sigma_{e e}/d \omega}$ are the differential interaction cross sections for each interaction \citep{Koch59,Haug75}, ${\rm n_e}$ and ${\rm n_Z}$ are the stationary electron and ion densities, and ${\rm v_e}$ is the electron velocity.

\subsection{Inverse Compton Scattering}\label{gamAstr:IC}

\ic{} scattering refers to the process by which a high-energy electron collides with a lower-energy photon transferring energy and ``upscattering" the photon to higher energies \citep{Blumenthal70}. 
There are several \isrfs{} available for upscattering by a population of electrons, such as that of the \cmb{}, with a temperature ${\rm T \approx 2.73~K}$,  \fir{} dust emission (${\rm T \approx 30~K}$), and \nir{} stellar-light (${\rm T \approx 3000~K}$). While the \cmb{} component is typically dominant  in the environs of an \snr{}, \cite{Porter06} showed that the other \isrfs{} can play a significant role, for example, in the inner Galaxy or when a star forming region is nearby. For a thermal photon-field of number density $n(\omega_i)$, with $\omega_i$ being the incident and photon energy.

\begin{equation}
n(\omega_i) = 
\frac{1}{\pi^2\lambda^3} 
\frac{\omega_i^2}{e^{\omega_i/\Theta} -1}
\end{equation}
where  $\Theta=kT/(m_e c^2)$, and the Compton wavelength of the electron is $\lambda=\hbar/(m_e c)$.
For a momentum distribution of relativistic electrons, ${\rm N_e(p)}$, embedded in an isotropic photon-field of number density $n(\omega_i)$, the single-photon differential photon-emissivity spectrum is:

\begin{equation}
{\rm \frac{dn}{d\omega dt} = 
c \int  n(\omega_i)d \omega_i
\int_{p_{min}}^\infty 
N_e(p)  \sigma_{KN}(\gamma,\omega_i,\omega)dp}
\end{equation}
with $\omega$ as the upscattered photon energy, ($\omega\equiv h\nu/(m_e c^2)$), and $\sigma_{KN}$ is the Klein-Nishina scattering  cross section:
\begin{equation}
{\rm \sigma_{KN}(\gamma,\omega_i,\omega) = \frac{2\pi r_0^2}{\omega_i \gamma^2}
\left[
1 + q - 2q^2 + 2q\ln q + \frac{\Gamma^2 q^2 (1-q)}{2(1+\Gamma q)}
\right]}
\end{equation}
and,
\begin{equation}
{\rm q \equiv \frac{\omega}{4 \omega_i \gamma (\gamma-\omega)}}
\end{equation}
for ${\rm \Gamma \equiv 4\omega_i \gamma}$, and the classical electron radius, ${\rm r_0 = e^2/(m_e c^2)}$.

%\jamie{We note that in the Thomson limit, ${\rm \omega_i \ll omega \ll \gamma mc^2}$ the Klein-Nishina cross section reduces to the Thomson cross section. }


\subsection{Neutral Pion Decay Emission}\label{gamAstr:PP}
When sufficiently high-energy protons ($\approx 280\mev{}$ \cite{Dermer13}) and ions collide with interstellar material, both charged and neutral pions are created (in about equal proportions) in the aftermath. The neutral pion subsequently (and expediently; within $~10^{-16}$ s) decays into two \gam{} photons (with a branching fraction of 98.8\% \cite{Beringer12}), each of rest energy $\omega_0 =(m_{\pi}c^2)/2\approx 67.5\mev{}$. For a differential proton distribution ${\rm N_p(p)}$, the differential photon distribution is:

\begin{equation}
\frac{d n}{d \omega d t} = 
n_p \int v_p N_p(p) 
\frac{d \sigma(p_\pi, p)}{d p}dp
\end{equation}
where ${\rm n_p}$ is the density of the target protons, ${\rm d \sigma(p_\pi, p)/dp}$ the differential cross section for neutral pion production for proton collisions, and ${\rm v_p}$ the non-thermal proton velocity \citep{Hillier84,Dermer86,Aharonian00}. The peak of the distribution occurs at $\approx 67.5\mev{}$ (which is half the rest mass of the neutral pion), however the peak is broadened due to Doppler shifting of the momentum distribution of the high-energy protons. This feature shows a bilateral symmetry about the peak energy in a photon spectrum representation, but in a ${\rm \nu F_\nu}$ representation, it appears as a hardening of the spectrum at a few hundred\mev{} \cite{Stecker71,Dermer13}. This characteristic pion-decay feature is colloquially referred to as the ``pion-bump", and is a vital instrument in distinguishing the proton-emitting population from the often-overlapping \ic{} and \brems{} spectrum produced at \gam{}s. 

%\section{Sources of \gam's}\label{gamAstr:Sources}
\jamie{]write about gamma ray sky, diffuse, talk about other catalogs! variety of sources pie chart sources we see
\twofgl{} \threefgl{} \onefhl{} \twopc{} tev pwn cat}


%\section{\gam{} Observations}\label{gamAstr:obs}

\jamie{when I get to \egret{}}
