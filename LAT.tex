\chapter{The \Fermi ~Gamma-Ray Space Telescope and \gam ~Data Analysis}
\label{chap:FGST}
\section{\label{sec:FGSTintro}Introduction}
The \Fermi~Gamma-Ray Space Telescope, successor to the \egret~instrument on \cgro, was successfully launched into orbit around Earth on June 11 2008. \Fermi ~consists of two instruments, the LAT and the \gbm.

 \jamie{maybe I don't need any gbm stuff? I mentioned it just to be complete about what fermi is }.  The LAT, which is the primary instrument on \Fermi, is a pair production telescope designed to detect photons from 20 \mev~to greater than 1 \tev. Its standard mode of operation is a sky-survey mode in which it observes the entire sky every 3 hours. The \gbm~is designed to detect \glspl{grb} in a waveband overlapping that of the \lat,  and complementary in lowering that energy range.  It is comprised of two types of scintillator detectors: two bismuth germanate crystals that operate from 150 \kev~ to 30 \mev, and 12 sodium iodide crystals sensitive to photons  between 8 \kev~to 1 \mev.

Combined the \lat~and \gbm~ make up a formidable observatory, spanning more than 8 decades in energy, and is  currently the only instrument performing all-sky observation in this broad energy range.


\section{\label{sec:FGSTLAT}The Large Area Telescope}

track and reconstruct the path of 
Describe it's objectives and strengths over predecessors, it is the only instrument currently operating in all sky mode in this broad energy range
Details on the LAT and it's design, be sure to focus on things that particularly pertain to the work I've done like what determines the PSF, thing about Pass 8 here maybe? Or maybe later on. 

what science was it designed to answer

general capabilities

details about aspect of the LAT related to extended sources, what determines PSF

Searching for extended Galactic sources leads into stellar remnants section?

Merged this with the LAT analysis  section

I'm not sure about this chapter yet. Maybe it's a general section on Analysis of Fermi data, why maximum likelihood, how it's formulated,  implemented in the Science Tools, pointlike and the analysis for extended sources. Diffuse emission

Four steps to going from observing the sky to final LAT analysis:

Instrument taking data: How we get to counts

Reconstruction : How we get photons

Likelihood: How to characterize sky using response functions, point source  and diffuse modeling

Likelihood for ES: how to use likelihood methods to char and resolve sources measure  extension


