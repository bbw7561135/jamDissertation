\phantomsection
\label{abstract}
%limit: 350 words (2450 characters)
When a massive star explodes as a supernova remnant it injects a huge amount of energy into its surroundings. The resultant expanding blast-wave is known as a supernova remnant. The shock created by the aftermath of the explosion is believed to be the primary accelerator of cosmic rays in our Galaxy. These supernova remnants are observable across the electro-magnetic spectrum. In 2008 the Fermi Gamma-Ray Space Telescope was launched into orbit and with its unprecedented gamma-ray resolution, has opened up a new window on the gamma-ray sky. In this thesis, we present new work using data from the primary instrument on the Fermi observatory, the Large Area Telescope (LAT), to study the population of supernova remnants observable by the LAT, with a focus on searching for spatially extended emission from these remnants of supernova explosions as a means for shedding light on the acceleration processes and gamma-ray emission mechanisms acting therein. 
