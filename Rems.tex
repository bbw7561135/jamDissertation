\chapter{Supernova Remnants: Theory and  Observation}
\label{chap:Rems}

\section{Introduction}\label{Rems:intro}
Why study SNRs, what they are, history of SNR,  radio detections, 



\section{\label{Rems:evo}Formation and Evolution}

-Stars die and explode, that energy is very quickly put into the surroundings
    
-snowplough, ST, radiative,
    
- what else?

How we detect gamma-rays from SNRs/PWNe in the Galaxy leads to and analysis section maybe?
\section{\label{Rems:obs}Morphology and Classification}

SNRs characterized by morphology and evolution properties

shell type, mixed morphology, filled center  composite )

Since I eventually do these all plane surveys, what does the spatial  distribution of them at radio look like?

Not sure how much to say about radio observations, x-ray, TeV

\section{\label{Rems:CR}Cosmic Ray SNR connection}

Give the whole, if 10\% of energy of SN explosion goes into particle acceleration, we can explain cosmic rays

Particle acceleration and DSA

This leads to gamma-ray section

	
\section{Summary}\label{Rems:summ} In this section we summarized the end phase of stellar evolution (just enough to motivate SNRs) and descried the environs surrounding the supernova; development and phases of \glspl{snr} (and \glspl{pwn}?).  In particular we detailed the nonthermal emission mechanisms that produce \g-ray radiation, detection of young vs middle-aged( evolved, interacting with surroundings/dense medium),TeV detects younger typically, the troubles of detecting extension from them(?) something about different emission zones? Troubles disentangling hadronic from leptonic at \g-rays. \g-ray spectral and morphological features. Trends across the population wrt spectral shape/breaks, higher luminosity for interacting rems. Cosmic rays, using gamma-rays to probe CR population. So much of \g-ray astro is really about studying CRs, how much to say about them? 

\section{Scratch}
This chapter needs a different title. It's more focused on the specific sources being studied in this thesis. Galactic extended sources, SNRs, PWNe, but as in the SNRcat, not just extended SNRs, point-like SNRs as well.

Less focus on PWNe. Only give as much as I feel I need to support mentioning them a bit for 2FHL?

The focus of this section is supernova remnants in a gamma-ray context. Theory of evolution, what the gamma-ray emission is like, what we can learn from them individually.  This leads to the 1st SNR cat section for what we can do with them ensemble

NOt sure I really need any PWN stuff yet

in 2FHL we detect some pwn. If including above 10gev work, they'll be there too. Much of the thesis is really about extended gamma-ray sources, but not sure how that fits into the title and chapters yet

Do I need to get into composite SNRs (composite means SNR + PWN ) Maybe relevant for G150? Some things about interaction of reverse shock with PWN and crushing/reverberations of the PWN?

\cite{Montmerle79}