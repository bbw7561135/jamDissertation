\begin{deluxetable}{cccccccc}
\setlength{\tabcolsep}{0.04in}
\tablewidth{0pt}
\tabletypesize{\scriptsize}
\tablecaption{LAT Analysis Results\label{tab:LATres}}
\tablehead{
\colhead{Spatial Model\,\tablenotemark{a}} & 
\colhead{TS} & 
\colhead{TS$_{ext}$} &
\colhead{$\sigma$} &
\colhead{Association} &
\colhead{Class} &
\colhead{Spatial model} &
\colhead{Extension [deg]} \\
\colhead{} & 
\colhead{} & 
\colhead{} &
\colhead{$^\circ$} &
\colhead{} &
\colhead{} &
\colhead{} &
\colhead{}}
\startdata
J0526.6$-$6825e      &    278.843 &    -32.850 & 49.80  & LMC                & gal    & 2D Gaussian         & 1.87 \\
J0617.2+2234e        &    189.048 &      3.033 & 398.64 & IC~443             & snr    & 2D Gaussian         & 0.27 \\
J0822.6$-$4250e      &    260.317 &  -3.277 &  63.87 & Puppis A       & snr    & ...               & 0.37 \\
%\cutinhead{Thee Point Sources}
%\sidehead{Uniform Disk}

\enddata
\tablecomments{~This is mostly from 2FHL, just playing with the table. Don't need a table for just disk hypothesis, but maybe to have disk 3  point sources compatison. I tried adding more sources on top of the 3FGL sources and there's no significant residual. where to say something about testing searching for point sources overlapping the extended source and trying to fit an extended source on top as well? in this table give the disk model with best spectral spatial params, TS, TSext  dof, LL, then the model with just the 3 3FGL sources (no disk) spectrum of each, TS, dof + LL( didn't relocalize the se sources), separate spatial spectral tables?
}
\tablenotetext{a}{comments and notes?}
\end{deluxetable}
