\chapter{Overview}
\label{chap:intro}

\begin{quote}
	``Maybe I'll have a super relevant quote here!'' 
	\begin{center}---by some awesome human, from \it{Some book} \end{center}
\end{quote}
%\begin{figure*}
%  \centering
%  \epsscale{1}
  % Jamie commented this out for now
 % \plotone{figures/filename-with-no-extension}
%  \caption[Short Caption]{Long caption}
%  \label{fig:somelabel}
%\end{figure*}

\section{Goooo $\gamma$-rays go!}
In this thesis we...
I think this should be super short. My abstract will introduce and motivate my work, this overview should just say here's the super short gist and here's the structure of this thesis? Maybe that's redundant with my preface?

Overview of the entire thesis, why gamma-rays, why the \lat{}, why \snrs{} and extended sources.

Higher energy studies with the LAT have been my focus since the beginning. Talk about what's nice about staying above 1 GeV, 10 GeV, 50 GeV. 

GeV TeV connection for 2FHL


egret was mostly pointed observation instrumented that would sometimes dwell on a spot for a couple of weeks, had a smaller field o view, didn't get as many photons (the LAT saw the entire EGRET sky in some short amount of time)

\snrs{} as sources of relativistic particles

Despite being the prime energy range to observe the effects of cosmic particle acceleration, the photon \sed{} resulting from these overlapping emission channels are often difficult to spectrally distinguish from one another.


With its unprecedented sensitivity and angular resolution above 1\gev, the \lat~provides for the first time the opportunity to distinguish SNR-emitted photons from their backgrounds, and  to unambiguously detect and identify dozens of \glspl{snr}. \jamie{maybe this is for the intro/abstract because it's a little vague}

The \lat{} is uniquely situated to address these goals and definitively detect and identify dozens \snrs

snrs as source of Galactic cosmic rays, and thus as drivers of Galactic evolution

\cite{Thomson93} gives the 68\% containment radius as $\theta \leq 5.85^\circ (E_\gamma /100 \mev )^{-0.534}$

\section{Dissertation Overview}



