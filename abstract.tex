\phantomsection
\label{abstract}
%limit: 350 words (2450 characters)
When a massive star explodes as a supernova,  it injects a huge amount of energy into its surroundings. The resultant expanding blast-wave and its interaction with the surrounding medium is known as a \snr{}. The shock created by the supernova event is believed to be the primary accelerator of \crs{} in our Galaxy. While \snrs{} are observable across the electromagnetic spectrum, studying the \gam{} emission from these sources is crucial in understanding the origin of \crs{} and acceleration processes acting therein. Recent advances in \gam{} astronomy present new opportunities to study the aftermath of stellar explosions at \gam{} energies. In 2008 the \Fermi{} Gamma-Ray Space Telescope was launched into orbit and, with its unmatched \gam{} resolution, has opened up a new window on the high-energy sky. In this thesis, I present new work using data from the primary instrument on the Fermi observatory, the \lat{}, to study both individual \snrs{} as well as the population of remnants observable by the LAT, with a focus on searching for spatially extended emission from these remnants.  To uniformly determine the high-energy properties of \snrs{}, I developed an automated method to systematically characterize the \gam{} emission in a region of the sky. Applying this method to the locations of several hundred radio-observed \snrs{}, we classified 30 \gam{} sources as likely being associated with \snrs{}. Our results, combined with archival radio, X-ray, and TeV observations, serve to challenge previously sufficient, simple \gam{} \snr{} emission models. I also present a study of the sources detected above 50 GeV, focusing on those lying in the Galactic plane.  31 sources were shown to be significantly spatially extended with 5 of those being newly detected. Finally, I present a dedicated analysis of one of the 5 newly detected extended sources. I determined that the extended GeV emission likely originated from the shock of  \snr{} \Gone{}. Combined with archival radio and X-ray data, I consider several possible origin scenarios, including one in which the \snr{} may be one of the youngest, closest \gam{} \snrs{} detected by the \lat{}.


