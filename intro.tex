\chapter{Introduction}
\label{chap:intro}

\begin{quote}
	``Maybe I'll have a super relevant quote here!'' 
	\begin{center}---by some awesome human, from \it{Some book} \end{center}
\end{quote}
%\begin{figure*}
%  \centering
%  \epsscale{1}
  % Jamie commented this out for now
 % \plotone{figures/filename-with-no-extension}
%  \caption[Short Caption]{Long caption}
%  \label{fig:somelabel}
%\end{figure*}

\section{Goooo $\gamma$-rays go!}

In this thesis we...or should I start with the extreme environs line?

Overview of the entire thesis, why gamma-rays, why the \lat, why \snr and \pwn and extended sources.

Higher energy studies with the LAT have been my focus since the beginning. Talk about what's nice about staying above 1 GeV, 10 GeV, 50 GeV. 

GeV TeV connection for 2FHL

Radio GeV for SNR cat (traces same particle population)

The advent of the \lat presents for the first time the capability to spectrally and spatially resolve \gls{snr} at \gev energies.

it is uniquely situated to address these issues

egret was mostly pointed observation instrumented that would sometimes dwell on a spot for a couple of weeks, had a smaller field o view, didn't get as many photons (the LAT saw the entire EGRET sky in some short amount of time)
\section{I Think I Hate Most of the Section Titles :(}

\section{Maybe None of the Chapters Need Introductions?}

\section{Dissertation Overview}


