scratch

\section{gammma}
Discovered by Paul Villard studying radiation from radium and named by Ernest Rutherford, who previously uncovered the nature of $\alpha$ and $\beta$ radiation, \gam{}s are the highest named energy of light 

more historical context 

what is a gamma-ray

why bother studying gamma-rays

probe of extreme environments

snrs as source of Galactic cosmic rays, and thus as drivers of Galactic evolution


LAT opened the sky greatly
Maybe this should just be a quick history of \gam{} telescopes leading up to fermi?
Quick rehash of method of detecting \gam's? Or is this just about previous \gam ~detectors and the state of the \gam~sky pre-Fermi?  mention telescopes up to EGRET, bit of detail on EGRET and what the pre-Fermi \gam~sky looked like, in particular in the context of SNRs, PWN, Galactic plane

gamma-ray telescopes leads into the LAT, Egret was  predecessor , what it did and what were some relevant unanswered questions regarding supernova remnants 


One of the primary goals of fermi was to identify these sources and the site of CR acceleration. But also to uniquely open this high energy window that where no other telescopes really operated
Brief history of radio detection of SNRs. detection at other wavelengths. what we see at \g-rays?

Motivation for why to study them

manifest
\citep{Sturner95}
\citep{Esposito96}

\section{Scratchgamma}

Altho' many miles from bomb zero, Dr. Bruce Banner is bathed in the full force of the mysterious Gamma Rays!

In this thesis we...
I think this should be super short. My abstract will introduce and motivate my work, this overview should just say here's the super short gist and here's the structure of this thesis? Maybe that's redundant with my preface?

Overview of the entire thesis, why gamma-rays, why the \lat{}, why \snrs{} and extended sources.

Higher energy studies with the LAT have been my focus since the beginning. Talk about what's nice about staying above 1 GeV, 10 GeV, 50 GeV. 

GeV TeV connection for 2FHL


egret was mostly pointed observation instrumented that would sometimes dwell on a spot for a couple of weeks, had a smaller field o view, didn't get as many photons (the LAT saw the entire EGRET sky in some short amount of time)

\snrs{} as sources of relativistic particles

Despite being the prime energy range to observe the effects of cosmic particle acceleration, the photon \sed{} resulting from these overlapping emission channels are often difficult to spectrally distinguish from one another.


With its unprecedented sensitivity and angular resolution above 1\gev, the \lat~provides for the first time the opportunity to distinguish SNR-emitted photons from their backgrounds, and  to unambiguously detect and identify dozens of \glspl{snr}. \jamie{maybe this is for the intro/abstract because it's a little vague}

The \lat{} is uniquely situated to address these goals and definitively detect and identify dozens \snrs

snrs as source of Galactic cosmic rays, and thus as drivers of Galactic evolution

\cite{Thomson93} gives the 68\% containment radius as $\theta \leq 5.85^\circ (E_\gamma /100 \mev )^{-0.534}$

\section{Scratch lat}


\jamie{Fermi goals: 1. Resolve the \gam~sky: the origins of diffuse emission and the nature of unidentified sources: Source identification through good source localization, measurement of spectra across broad energy range, nearly continuous monitoring of the sky for temporal variability 2. Understand the mechanisms of particle acceleration in celestial sources}

\jamie{don't need this I think. The \gbm{} is designed to detect \grb{}s in a waveband overlapping that of the \lat{} yet complementary in that its energy extends considerably lower.  It is comprised of two types of scintillator detectors: two bismuth germanate crystals that operate from 150 \kev{} to 30 \mev{}, and 12 sodium iodide crystals sensitive to photons  between 8 \kev{} and 1 \mev{}. }

\section{ScratchRems}

\jamie{2FGL only had 7 ID'd SNR, 4 snr, 58 spp. 3FGL had 12 SNR, 11 snr, 49 spp. I'm not sure what made some snr vs. spp. They must have all been point sources right? No known radio pwn, psr?}

\jamie{something about TeV observations. Predominantly young or parts of interacting rems (like the signs of escape) 12 shell type, 1 composite 7 SNR/MC. see https://arxiv.org/pdf/1508.05190v1.pdf or https://arxiv.org/pdf/1510.01373v1.pdf}

\section{scratch snrcat}
\jamie{EGRET point source sensitivity is ${\rm \sim1x10^{-7}}\flux{}$
	\url{http://fermi.gsfc.nasa.gov/science/instruments/table1-1.html} get this number from some paper instead?}

\jamie{Another pointlike assumption to speed things up is that the PSF doesn't vary too much with event incidence angle in individual bins. To ensure this even more, events with a reconstructed angle > 66.4deg (cos theta = 0.4) are removed (idk why this angle)}

\jamie{I did work for mock catalog, but it was really just running addsrcs centered on the mock positions}

\section{G150}

${\rm L_{\gamma} = 1.3 \times 10^{33}~erg~s^{-1}}$ from 1 GeV to 1 TeV for best d and flux above


energy flux from 100 MeV to 100 GeV: ${\rm 4.84 \times 10^{-11}~erg~cm^{-2}~s^{-1}} $

${\rm L_{\gamma} = 8.6 \times 10^{32}~erg~s^{-1}}$ from 100 MeV to 100 GeV for best d and flux in same range

energy flux from 1 GeV to 100 GeV: ${\rm 3.83 \times 10^{-11}~erg~cm^{-2}~s^{-1}} $

${\rm L_{\gamma} = 6.8 \times 10^{32}~erg~s^{-1}}$ from 1 geV to 100 GeV for best d and flux in same range

For diamMax = 60pc, dmax = 1.22kpc, and Lmax (100mev-100GeV) = 8.7e+33

\jamie{One potential scenario is that the \gam{}s in this region are produced by a pulsar wind nebula (PWN) generated by the putative pulsar of SNR \Gone{}.} 

\jamie{for synchrotron $\alpha = (1-s)/2$, where $\alpha$ is the radio spectral index, and s the electron distribution power law index. Same for IC below break?}

\jamie{SNR cat figure 8 suggests there are only 4(ish) SNRs with an index less than 2}

\jamie{eastern shell (Jack called this overall) radio index $\alpha = 0.4 \pm 0.17$  \cite{Gao14}, but $\alpha = 0.69 \pm 0.24$ for the western}

\jamie{For energies below the high energy break, For pion and bremss $\Gamma = 2\alpha + 1$ (says SNR cat) For IC, $\Gamma = \alpha + 1 $ ) for positive $\alpha$}

\jamie{From \cite{Gaensler06} Typical indices for PWNe are $\sim -0.3 \lesssim \alpha  \lesssim  0$ in the radio band, and ($\Gamma \approx 2$) in the X-ray band. So $\alpha$ is not inconsistent, but at the boundary.}

\jamie{For puppis A paper, why did they use particle index = gam photon index?}

\jamie{cr abundances at earth kep = 0.01  (Hillas 2005). I think in the puppis A paper the use 0.02?}

\jamie{Sooo, my index is consistent with either?}

\jamie{how off can dist be? SNR catalog says 8 SNRs are withing 1.5 kpc and have some kind of classification in the catalog. These are (closest first) Vela, cygnus loop , Vela Jr, RX J1713, G073, S147, IC443, Monoceros loop. if 0.4 kpc is correct for G150, it's the second closest LAT detected SNR. Even at 1.5 kpc it would be within top 10}