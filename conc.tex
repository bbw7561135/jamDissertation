\chapter{Conclusions}\label{chap:conc}
In this dissertation we investigated the \gam{} emission from \snrs{} to obtain a better understanding of the origin of Galactic \crs{}. We performed both individual and population studies of the nonthermal, high-energy, radiation from \snrs{} as a means of probing the underlying \cray{} particle population. The advent of the \lat{}, onboard the \Fermi{} Gamma-Ray Space Telescope, has proved a great boon to studying the \snr{}/\cray{} connection. The \lat{}'s excellent spatial and spectral resolution combined with its wide field of view and survey mode operation has provided for the first time the opportunity to study the \gam{} morphology of \snrs{} as well as probe the hadronic and leptonic \cray{} acceleration processes generating \gam{} photons.

One part of this thesis consisted of the development of an automated analysis method, rooted in a maximum-likelihood framework, to systematically detect and add statistically significant \gam{}-ray sources to an \roi{}. Applying this method, we performed for the first time a characterization of the\gev{} emission around all radio observed \snrs{} with the goal of uniformly cataloging the global properties of this \gam{} emitting population. This analysis allowed us to classify 30 sources as likely\gev{}  \snrs{} and we found that previously adequate, simple \cray{} acceleration models can not sufficiently describe the spectral properties of many \lat{} observed \snrs{}. 

Next, we conducted a survey of the \gam{} sky between 50\gev{} and 2\tev{}, making use of the recent pass 8 event reconstruction analysis. This work served to bridge the energy gap between previous \lat{} 1-100}\gev{} catalog observations and those of ground-based, Cherenkov telescopes with operating energies extending to the sub-TeV regime. A total of 360 sources were detected across the entire sky, with only 25\% previously detected by \iacts{}, demonstrating that this catalog can serve to improve the efficiency of\tev{} source detection and act as pathfinder for future \iact{} observations. The work presented in this dissertation focused on the Galactic results of the catalog, with 11\% of sources being associated with known Galactic objects (primarily \snrs{} and \pwne{}), and 13\% having no known multiwavelength counterpart. We showed that the Galactic objects display significantly harder spectra than extragalactic sources, suggesting that many of the unidentified sources are likely of a Galactic origin. 31 sources were detected as spatially-extended. 4 of these were previously detected but unresolved by the \lat{}, and one, partially overlapping SNR \Gone{}, was the first blindly-detected extended \lat{} source. 

The final study presented here is a follow-up analysis of the \gam{} emission coincident with SNR \Gone{} aimed at understanding the origin of the extended \gam{} source and its connection to the radio \snr{}. Lowering the analysis energy range down to 1\gev{}, we detected significantly extended,  hard \gam{} emission with a morphology in excellent agreement with the large, radio observed \snr{}. We employed archival HI and X-ray data to estimate the distance to \Gone{}  and constrain the ambient density in the vicinity of the remnant. Combining our\gev{} results the with these environmental constraints, we determined that the \snr{} is between 500  and a few thousand years old, likely placing \Gone{} in the dynamically young stage of evolution \ie{} not yet interacting with surrounding molecular material. Supporting this statement, we also determined that the spectral properties and under-luminous power emitted by the \snr{} in \gam{}s is inconsistent with an evolved \snr{} whose shock has encountered nearby molecular clouds. 

In closing, these studies demonstrate the ongoing potential of the \lat{} to reveal and fill in the gaps in the faint radio and \gam{} emitting \snr{} population, particularly in the higher energy, signal-dominated (\ie{} sensitivity improves linearly with time) regime. The work performed in this dissertation, along with future \lat{} \snr{} studies will be of tantamount importance for the up and coming class of ground-based Cherenkov telescopes. Follow-up pointed observations of the  catalog sources, first detected and presented in this dissertation, with\tev{} telescopes (\veritas{}, \hess{}, and the forthcoming Cherenkov Telescope Array) will be able to resolve structure on a finer scale, allowing for refined analysis of the \gam{} emission mechanisms acting at \snr{} shock fronts. \hawc{} observatory, with it's wide field of view, and \textit{}overhead sky-survey method will complement the study of the sources detected in this thesis, with both an overlapping energy range and the capability to detect broadly extended sources. Leveraging the \lat{}'s unique capabilities in tandem with broadband observations of \snrs{} will aid in uncovering and resolving these \gam{} sources which is vital to assessing the ability of \snrs{} to account for the total power in \crs{} in our Galaxy, and to determine if \snrs{} can truly accelerate particles to the knee of the \cray{} energy spectrum. 





