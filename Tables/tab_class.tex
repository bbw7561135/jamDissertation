% Table listing the source classes and their numbers
\begin{deluxetable}{lcr}
\setlength{\tabcolsep}{0.04in}
\tablewidth{0pt}
\tabletypesize{\small}
\tcap{2FHL Source Classes \label{tab:class}}
\tablehead{
\colhead{Description} & 
\multicolumn{2}{c}{Associated} \\
& 
\colhead{Designator} &
\colhead{Number}
}
\startdata
Pulsar & psr & 1 \\
Pulsar wind nebula & pwn & 14 \\
Supernova remnant & snr & 16 \\
Supernova remnant / Pulsar wind nebula & spp & 4 \\
High-mass binary & hmb & 2 \\
Binary & bin & 1 \\
Star-forming region & sfr & 1 \\
BL Lac type of blazar & bll & 180 \\
BL Lac type of blazar with prominent galaxy emission & bll-g & 13 \\
FSRQ type of blazar & fsrq & 10 \\
Non-blazar active galaxy & agn & 2 \\ 
Radio galaxy & rdg & 4 \\
Radio galaxy / BL Lac  & rdg/bll & 2 \\
Blazar candidate of uncertain type I & bcu I & 7 \\
Blazar candidate of uncertain type II & bcu II & 34 \\ 
Blazar candidate of uncertain type III & bcu III & 19 \\  
Normal galaxy (or part) & gal & 1 \\
Galaxy cluster & galclu & 1 \\
Total associated & \nodata & 312 \\
%\hline
Unassociated & \nodata & 48 \\ 
\hline
Total in 2FHL & \nodata & 360 \\ 

\enddata
\tablecomments{The designation `spp' indicates potential association with SNR or PWN.
The `bcu I', `bcu II', and `bcu III' classes are derived from 3LAC  and describe the increasing lack of multiwavelength information to classify  the source as a blazar \citep[see ][for more details]{3LAC}. The designation `bll-g' is adapted from the BZCAT \citep{bzcat5} and indicates a blazar whose SED has a significant contribution from the host galaxy.}
\end{deluxetable}
