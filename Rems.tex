\chapter{Stellar Remnants}
\label{chap:Rems}
Is it weird to call them stellar remnants? It's not like the remnant of the star, but of the supernova
\section{Introduction}\label{Rems:intro}
Brief history of radio detection of SNRs. detection at other wavelengths. what we see at \g-rays?
\section{Supernova Remnants}
History of SNR,  radio detections, then x-ray, can't really ID in new ones with the LAT, theory: snow plow , ST evolution, radiative phase, shell type, mixed morphology (shell plus center filled thermal x-ray) 
\section{Pulsar Wind Nebulae}
How much of my thesis is really about pwn? in 2FHL we detect some. if including above 10gev work, they'll be there too. Much of the thesis is really about extended gamma-ray sources, but not sure how that fits into the title and chapters yet

Do I need to get into composite SNRs (I think composite means SNR + PWN right?) Maybe relevant for G150? Some things about interaction of reverse shock with PWN and crushing/reverberations of the PWN?

How we detect gamma-rays from SNRs/PWNe in the Galaxy leads to and analysis section maybe?

\section{Summary}\label{Rems:summ} In this section we summarized the end phase of stellar evolution (just enough to motivate SNRs) and descried the environs surrounding the supernova; development and phases of \glspl{snr} (and \glspl{pwn}?).  In particular we detailed the nonthermal emission mechanisms that produce \g-ray radiation, detection of young vs middle-aged( evolved, interacting with surroundings/dense medium),TeV detects younger typically, the troubles of detecting extension from them(?) something about different emission zones? Troubles disentangling hadronic from leptonic at \g-rays. \g-ray spectral and morphological features. Trends across the population wrt spectral shape/breaks, higher luminosity for interacting rems. Cosmic rays, using gamma-rays to probe CR population. So much of \g-ray astro is really about studying CRs, how much to say about them? 