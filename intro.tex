\chapter{Introduction}
\label{chap:intro}

\begin{quote}
	``Maybe I'll have a super relevant quote here!'' 
	\begin{center}---by some awesome human, from \it{Some book} \end{center}
\end{quote}
%\begin{figure*}
%  \centering
%  \epsscale{1}
  % Jamie commented this out for now
 % \plotone{figures/filename-with-no-extension}
%  \caption[Short Caption]{Long caption}
%  \label{fig:somelabel}
%\end{figure*}

\section{Goooo $\gamma$-rays go!}

In this thesis we...or should I start with the extreme environs line?

Overview of the entire thesis, why gamma-rays, why the \lat, why \snr and \pwn and extended sources.

Higher energy studies with the LAT have been my focus since the beginning. Talk about what's nice about staying above 1 GeV, 10 GeV, 50 GeV. 

GeV TeV connection for 2FHL

Radio GeV for SNR cat (traces same particle population)


\section{Dissertation Overview}