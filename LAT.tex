\chapter{The \Fermi ~Gamma-Ray Space Telescope and \gam ~Data Analysis}
\label{chap:FGST}

\section{\label{FGST:intro}Introduction}
The \Fermi~Gamma-Ray Space Telescope (\Fermi ~hereafter), successor to the \egret~instrument on \cgro, was successfully launched into orbit around Earth on June 11 2008. \Fermi ~consists of two instruments, the LAT and the \gbm.

The LAT, which is the primary instrument on \Fermi, is a pair conversion telescope designed to detect photons from 20 \mev~to greater than 1 \tev. Its standard mode of operation is a sky-survey mode in which it observes the entire sky every 3 hours. The \gbm~is designed to detect \glspl{grb} in a waveband overlapping that of the \lat,  and complementary in lowering that energy range.  It is comprised of two types of scintillator detectors: two bismuth germanate crystals that operate from 150 \kev~ to 30 \mev, and 12 sodium iodide crystals sensitive to photons  between 8 \kev~and 1 \mev. 

Combined the \lat~and \gbm~ make up a formidable observatory, spanning more than 8 decades in energy, and is  currently the only instrument performing all-sky observation in this broad energy range.  \jamie{maybe I don't need any gbm stuff? I mentioned it just to be complete about what fermi is, probably won't mention it again, and this last par doesn't really flow into the next }


\section{\label{FGST:LAT}The Large Area Telescope}
The need for Fermi in the context of what EGRET did

What were open questions from EGRET era, state of \gam~detection of SNRs, what question was Fermi deigned to answe

Description of the instrument
 \jamie{not sure this really goes here, separate section for what questions  Fermi was designed to answer?}
 
 
 track and reconstruct the path of 
 Describe it's objectives and strengths over predecessors
 Details on the LAT and it's design, be sure to focus on things that particularly pertinent to the work I've done like what determines the PSF, thing about Pass 8 here maybe? Or maybe later on. 
 
 what science was it designed to answer
 
 general capabilities
 
 details about aspect of the LAT related to extended sources, what determines PSF
 \section{\label{FGST:analysis}\gam~Data Analysis}
 
 Why maximum likelihood, how it's formulated,  implemented in the Science Tools, pointlike and the analysis for extended sources. Diffuse emission.
 
 
 Four steps to going from observing the sky to final LAT analysis:
 
 Instrument taking data: How we get to counts
 
 Reconstruction : How we get photons
 
 Likelihood: How to characterize sky using response functions, point source  and diffuse modeling
 
 Likelihood for ES: how to use likelihood methods to char and resolve sources measure  extension
 
 Section on diffuse emission
\subsection{\label{FGST:sub}Do I need subsections?}
\section{Scratch}


\jamie{ Fermi goals: 1. Resolve the \gam~sky: the origins of diffuse emission and the nature of unidentified sources: Source identification through good source localization, measurement of spectra across broad energy range, nearly continuous monitoring of the sky for temporal variability 2. Understand the mechanisms of particle acceleration in celestial sources}
I'm not sure about this chapter yet. Maybe it's a general section on Analysis of Fermi data,


